\documentclass[]{article}
\usepackage{lmodern}
\usepackage{amssymb,amsmath}
\usepackage{ifxetex,ifluatex}
\usepackage{fixltx2e} % provides \textsubscript
\ifnum 0\ifxetex 1\fi\ifluatex 1\fi=0 % if pdftex
  \usepackage[T1]{fontenc}
  \usepackage[utf8]{inputenc}
\else % if luatex or xelatex
  \ifxetex
    \usepackage{mathspec}
  \else
    \usepackage{fontspec}
  \fi
  \defaultfontfeatures{Ligatures=TeX,Scale=MatchLowercase}
\fi
% use upquote if available, for straight quotes in verbatim environments
\IfFileExists{upquote.sty}{\usepackage{upquote}}{}
% use microtype if available
\IfFileExists{microtype.sty}{%
\usepackage{microtype}
\UseMicrotypeSet[protrusion]{basicmath} % disable protrusion for tt fonts
}{}
\usepackage[margin=1in]{geometry}
\usepackage{hyperref}
\hypersetup{unicode=true,
            pdftitle={Sandbox\_R\_testCompile.R},
            pdfauthor={markloessi},
            pdfborder={0 0 0},
            breaklinks=true}
\urlstyle{same}  % don't use monospace font for urls
\usepackage{color}
\usepackage{fancyvrb}
\newcommand{\VerbBar}{|}
\newcommand{\VERB}{\Verb[commandchars=\\\{\}]}
\DefineVerbatimEnvironment{Highlighting}{Verbatim}{commandchars=\\\{\}}
% Add ',fontsize=\small' for more characters per line
\usepackage{framed}
\definecolor{shadecolor}{RGB}{248,248,248}
\newenvironment{Shaded}{\begin{snugshade}}{\end{snugshade}}
\newcommand{\KeywordTok}[1]{\textcolor[rgb]{0.13,0.29,0.53}{\textbf{#1}}}
\newcommand{\DataTypeTok}[1]{\textcolor[rgb]{0.13,0.29,0.53}{#1}}
\newcommand{\DecValTok}[1]{\textcolor[rgb]{0.00,0.00,0.81}{#1}}
\newcommand{\BaseNTok}[1]{\textcolor[rgb]{0.00,0.00,0.81}{#1}}
\newcommand{\FloatTok}[1]{\textcolor[rgb]{0.00,0.00,0.81}{#1}}
\newcommand{\ConstantTok}[1]{\textcolor[rgb]{0.00,0.00,0.00}{#1}}
\newcommand{\CharTok}[1]{\textcolor[rgb]{0.31,0.60,0.02}{#1}}
\newcommand{\SpecialCharTok}[1]{\textcolor[rgb]{0.00,0.00,0.00}{#1}}
\newcommand{\StringTok}[1]{\textcolor[rgb]{0.31,0.60,0.02}{#1}}
\newcommand{\VerbatimStringTok}[1]{\textcolor[rgb]{0.31,0.60,0.02}{#1}}
\newcommand{\SpecialStringTok}[1]{\textcolor[rgb]{0.31,0.60,0.02}{#1}}
\newcommand{\ImportTok}[1]{#1}
\newcommand{\CommentTok}[1]{\textcolor[rgb]{0.56,0.35,0.01}{\textit{#1}}}
\newcommand{\DocumentationTok}[1]{\textcolor[rgb]{0.56,0.35,0.01}{\textbf{\textit{#1}}}}
\newcommand{\AnnotationTok}[1]{\textcolor[rgb]{0.56,0.35,0.01}{\textbf{\textit{#1}}}}
\newcommand{\CommentVarTok}[1]{\textcolor[rgb]{0.56,0.35,0.01}{\textbf{\textit{#1}}}}
\newcommand{\OtherTok}[1]{\textcolor[rgb]{0.56,0.35,0.01}{#1}}
\newcommand{\FunctionTok}[1]{\textcolor[rgb]{0.00,0.00,0.00}{#1}}
\newcommand{\VariableTok}[1]{\textcolor[rgb]{0.00,0.00,0.00}{#1}}
\newcommand{\ControlFlowTok}[1]{\textcolor[rgb]{0.13,0.29,0.53}{\textbf{#1}}}
\newcommand{\OperatorTok}[1]{\textcolor[rgb]{0.81,0.36,0.00}{\textbf{#1}}}
\newcommand{\BuiltInTok}[1]{#1}
\newcommand{\ExtensionTok}[1]{#1}
\newcommand{\PreprocessorTok}[1]{\textcolor[rgb]{0.56,0.35,0.01}{\textit{#1}}}
\newcommand{\AttributeTok}[1]{\textcolor[rgb]{0.77,0.63,0.00}{#1}}
\newcommand{\RegionMarkerTok}[1]{#1}
\newcommand{\InformationTok}[1]{\textcolor[rgb]{0.56,0.35,0.01}{\textbf{\textit{#1}}}}
\newcommand{\WarningTok}[1]{\textcolor[rgb]{0.56,0.35,0.01}{\textbf{\textit{#1}}}}
\newcommand{\AlertTok}[1]{\textcolor[rgb]{0.94,0.16,0.16}{#1}}
\newcommand{\ErrorTok}[1]{\textcolor[rgb]{0.64,0.00,0.00}{\textbf{#1}}}
\newcommand{\NormalTok}[1]{#1}
\usepackage{graphicx,grffile}
\makeatletter
\def\maxwidth{\ifdim\Gin@nat@width>\linewidth\linewidth\else\Gin@nat@width\fi}
\def\maxheight{\ifdim\Gin@nat@height>\textheight\textheight\else\Gin@nat@height\fi}
\makeatother
% Scale images if necessary, so that they will not overflow the page
% margins by default, and it is still possible to overwrite the defaults
% using explicit options in \includegraphics[width, height, ...]{}
\setkeys{Gin}{width=\maxwidth,height=\maxheight,keepaspectratio}
\IfFileExists{parskip.sty}{%
\usepackage{parskip}
}{% else
\setlength{\parindent}{0pt}
\setlength{\parskip}{6pt plus 2pt minus 1pt}
}
\setlength{\emergencystretch}{3em}  % prevent overfull lines
\providecommand{\tightlist}{%
  \setlength{\itemsep}{0pt}\setlength{\parskip}{0pt}}
\setcounter{secnumdepth}{0}
% Redefines (sub)paragraphs to behave more like sections
\ifx\paragraph\undefined\else
\let\oldparagraph\paragraph
\renewcommand{\paragraph}[1]{\oldparagraph{#1}\mbox{}}
\fi
\ifx\subparagraph\undefined\else
\let\oldsubparagraph\subparagraph
\renewcommand{\subparagraph}[1]{\oldsubparagraph{#1}\mbox{}}
\fi

%%% Use protect on footnotes to avoid problems with footnotes in titles
\let\rmarkdownfootnote\footnote%
\def\footnote{\protect\rmarkdownfootnote}

%%% Change title format to be more compact
\usepackage{titling}

% Create subtitle command for use in maketitle
\providecommand{\subtitle}[1]{
  \posttitle{
    \begin{center}\large#1\end{center}
    }
}

\setlength{\droptitle}{-2em}

  \title{Sandbox\_R\_testCompile.R}
    \pretitle{\vspace{\droptitle}\centering\huge}
  \posttitle{\par}
    \author{markloessi}
    \preauthor{\centering\large\emph}
  \postauthor{\par}
      \predate{\centering\large\emph}
  \postdate{\par}
    \date{2019-05-12}


\begin{document}
\maketitle

\begin{Shaded}
\begin{Highlighting}[]
\CommentTok{# sandbox on use of in R}
\CommentTok{# https://rmarkdown.rstudio.com/articles_report_from_r_script.html}
\CommentTok{# file > compile report}

\CommentTok{# install.packages('quantmod')}
\KeywordTok{library}\NormalTok{(quantmod)}
\end{Highlighting}
\end{Shaded}

\begin{verbatim}
## Loading required package: xts
\end{verbatim}

\begin{verbatim}
## Loading required package: zoo
\end{verbatim}

\begin{verbatim}
## 
## Attaching package: 'zoo'
\end{verbatim}

\begin{verbatim}
## The following objects are masked from 'package:base':
## 
##     as.Date, as.Date.numeric
\end{verbatim}

\begin{verbatim}
## Loading required package: TTR
\end{verbatim}

\begin{verbatim}
## Version 0.4-0 included new data defaults. See ?getSymbols.
\end{verbatim}

\begin{Shaded}
\begin{Highlighting}[]
\CommentTok{# We'll use Yahoo here but Google Finance is also supported. }
\CommentTok{# https://finance.yahoo.com/lookup/}

\CommentTok{# We'll call multiple symbols at once by creating a vector of symbols}
\CommentTok{# and then pass the basket to the getSymbols function.}

\NormalTok{symbolBasket <-}\StringTok{ }\KeywordTok{c}\NormalTok{(}\StringTok{'AAPL'}\NormalTok{, }\StringTok{'AMZN'}\NormalTok{, }\StringTok{'BRK-B'}\NormalTok{, }\StringTok{'SPY'}\NormalTok{)}
\KeywordTok{getSymbols}\NormalTok{(symbolBasket , }\DataTypeTok{src=}\StringTok{'yahoo'}\NormalTok{)}
\end{Highlighting}
\end{Shaded}

\begin{verbatim}
## 'getSymbols' currently uses auto.assign=TRUE by default, but will
## use auto.assign=FALSE in 0.5-0. You will still be able to use
## 'loadSymbols' to automatically load data. getOption("getSymbols.env")
## and getOption("getSymbols.auto.assign") will still be checked for
## alternate defaults.
## 
## This message is shown once per session and may be disabled by setting 
## options("getSymbols.warning4.0"=FALSE). See ?getSymbols for details.
\end{verbatim}

\begin{verbatim}
## [1] "AAPL"  "AMZN"  "BRK-B" "SPY"
\end{verbatim}

\begin{Shaded}
\begin{Highlighting}[]
\KeywordTok{lineChart}\NormalTok{(}\StringTok{`}\DataTypeTok{BRK-B}\StringTok{`}\NormalTok{, }\DataTypeTok{line.type =} \StringTok{'h'}\NormalTok{, }\DataTypeTok{theme =} \StringTok{'white'}\NormalTok{)}
\end{Highlighting}
\end{Shaded}

\includegraphics{Sandbox_R_testCompile_files/figure-latex/unnamed-chunk-1-1.pdf}

\begin{Shaded}
\begin{Highlighting}[]
\CommentTok{# barChart(`BRK-B`, bar.type = 'hlc', TA = NULL)}
\CommentTok{# }
\CommentTok{# candleChart(`BRK-B`, TA=NULL, subset = '2019')}
\CommentTok{# candleChart(AAPL, TA=NULL, subset = '2019')}
\CommentTok{# candleChart(`BRK-B`, TA=c(addMACD()), subset = '2019')}
\CommentTok{# }
\CommentTok{# chartSeries(`BRK-B`, }
\CommentTok{#             type = c("auto", "matchsticks"), }
\CommentTok{#             subset = '2018-01::',}
\CommentTok{#             show.grid = TRUE,}
\CommentTok{#             major.ticks='auto', minor.ticks=TRUE,}
\CommentTok{#             multi.col = TRUE,}
\CommentTok{#             TA=c(addMACD(),addVo()))}
\CommentTok{# }
\CommentTok{# BRKB <- as.xts(`BRK-B`)}
\CommentTok{# names(BRKB)}
\CommentTok{# names(BRKB) <- c("BRKB.Open"   ,  "BRKB.High"   ,  "BRKB.Low"   ,   "BRKB.Close"  ,  "BRKB.Volume",  "BRKB.Adjusted")}
\CommentTok{# names(BRKB)}
\CommentTok{# }
\CommentTok{# BRKB.EMA.20<- EMA(BRKB$BRKB.Close, n=20)}
\CommentTok{# BRKB.EMA.50<- EMA(BRKB$BRKB.Close, n=50)}
\CommentTok{# BRKB.EMA.100<- EMA(BRKB$BRKB.Close, n=100)}
\CommentTok{# BRKB.EMA.200<- EMA(BRKB$BRKB.Close, n=200)}
\CommentTok{# }
\CommentTok{# chartSeries(`BRK-B`, }
\CommentTok{#             type = c("auto", "matchsticks"), }
\CommentTok{#             subset = '2018-01::',}
\CommentTok{#             show.grid = TRUE,}
\CommentTok{#             major.ticks='auto', minor.ticks=TRUE,}
\CommentTok{#             multi.col = FALSE,}
\CommentTok{#             TA=c(addMACD(),addVo()))}
\CommentTok{# }
\CommentTok{#             addTA(BRKB.EMA.20, on=1, col = "green")}
\CommentTok{#             addTA(BRKB.EMA.50, on=1, col = "blue")}
\CommentTok{#             addTA(BRKB.EMA.100, on=1, col = "yellow")}
\CommentTok{#             addTA(BRKB.EMA.200, on=1, col = "red")}
\CommentTok{# }
\end{Highlighting}
\end{Shaded}


\end{document}
